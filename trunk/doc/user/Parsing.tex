\section{Parsing}
\label{sec:Parsing}

The parsing interface of EXIP is similar to SAX and StAX. The main difference is
that most of the XML schema build-in types are passed in a native binary form
rather than as a string representation.
The EXIP parser processes the EXI stream in one direction and produces events
on occurrence of information items such as elements, attributes or content values.
The applications using EXIP must register callback functions for the events that they are interested in.
In this way the data from the EXI stream is delivered to the application as a
parameters to these callback functions.
The header \texttt{contentHandler.h} provides declarations of the callback functions
while the API for controlling the progress through the EXI stream is in \texttt{EXIParser.h}.

\subsection{Schema-less decoding}

When in schema-less mode all value items are encoded with strings and hence
delivered to the application through the \texttt{stringData()} callback handler.
Similarly to the serialization, the parsing consists of 7 simple steps:
\begin{enumerate}
 \item Declare a parser container that holds the serialized data and EXIP state:
\begin{lstlisting}
Parser parser;
\end{lstlisting}

 \item Define a binary input buffer for reading the EXI stream.
The buffer can optionally include an external input stream for fetching the
serialized EXI data. If an input stream is not defined the whole EXI
document must be stored in memory before starting the
parsing process. The following code declares and initializes an input memory buffer:
\begin{lstlisting}
BinaryBuffer buffer;
char buf[INPUT_BUFFER_SIZE];

buffer.buf = buf;
buffer.bufLen = INPUT_BUFFER_SIZE;
buffer.bufContent = 0;
\end{lstlisting}
Beside these definitions, when a file is used to fetch the
EXI data the following steps are also required: 
\begin{lstlisting}
/* IN CASE OF FILE INPUT STREAM */
/* Define a function implementing the actual reading from a file */
size_t readFromFileInputStream(void* buf, size_t readSize, void* stream)
{
	FILE *infile = (FILE*) stream;
	return fread(buf, 1, readSize, infile);
}

FILE *infile; /* Using a file for fetching the EXI data */
infile = fopen(sourceEXIFile, "rb" ); /* open the file before use */
buffer.ioStrm.readWriteToStream = readFromFileInputStream;
buffer.ioStrm.stream = infile; /* Sets the input stream to the file */
\end{lstlisting}
If no file or other input streams are being used (e.i. the whole EXI message is
stored in \texttt{buffer.buf}), the input function and the stream
of the \texttt{BinaryBuffer} must be initialized with NULL values:
\begin{lstlisting}
/* IN CASE OF IN-MEMORY ONLY INPUT*/
buffer.ioStrm.readWriteToStream = NULL;
buffer.ioStrm.stream = NULL;
\end{lstlisting}

The size of the \texttt{INPUT\_BUFFER\_SIZE} macro depends on the
use case. When in in-memory input mode or sufficient RAM on the
platform the buffer size should be kept high. Note that even if
an input stream is used such as a file, a non-zero length buffer
for temporary storing parts of the EXI data is still needed.

 \item Define application data and initialize the parser object:
\begin{lstlisting}
/** The application data that is passed to the callback handlers*/
struct applicationData
{
  unsigned int elementCount;
  unsigned int nestingLevel;
} appData;

/**
 * @param[in, out] parser EXIP parser container
 * @param[in] buffer binary buffer for fetching EXI encoded data
 * @param[in] NULL a compiled schema information to be used for schema
		      enabled processing; NULL if no schema is available
 * @param[in, out] appData application data to be passed to the callback handlers
 */  
initParser(&parser, buffer, NULL, &appData);    
\end{lstlisting}

 \item Initialize the application data and register the callback handlers with the parser object:
\begin{lstlisting}
appData.elementCount = 0; /* Example: the number of elements passed */
appData.nestingLevel = 0; /* Example: the nesting level */

parser.handler.fatalError = sample_fatalError;
parser.handler.error = sample_fatalError;
parser.handler.startDocument = sample_startDocument;
parser.handler.endDocument = sample_endDocument;
parser.handler.startElement = sample_startElement;
parser.handler.attribute = sample_attribute;
parser.handler.stringData = sample_stringData;
parser.handler.endElement = sample_endElement;

/** According to the above definitions:
  *  When the parser start parsing the body, the sample_startDocument()
  *  callback will be invoked; when a start of an element is parsed the
  *  sample_startElement() will be invoked and so on. All of the events
  *  for which there is no handler registered will be discarded. 
  */
\end{lstlisting}

\textbf{NOTE:} If the EXI options are communicated with an out-of-band mechanism and are
not included in the EXI header of the stream they MUST be set here
(see Section \ref{sec:Options} for more information on how to set up
the EXI options).

 \item Parse the header of the stream:
\begin{lstlisting}
parseHeader(&parser);
/* The header fields are stored in parser.strm.header*/
\end{lstlisting}

 \item Parse the body of the EXI stream, one content item at a time:
\begin{lstlisting}
errorCode error_code = ERR_OK;
while(error_code == ERR_OK)
{
  error_code = parseNext(&parser);
}

/**
 * On successful parsing step, the parseNext() returns ERR_OK if there
 * are more content items left for parsing and PARSING_COMPLETE in case
 * the parsing is complete. If error conditions occur during the
 * process it returns an error code.
 */
\end{lstlisting}

 \item Destroy the parser object and free the memory allocated by it. If any other
streams are left open close them as well:
\begin{lstlisting}
destroyParser(&parser);
fclose(infile);
\end{lstlisting}

\end{enumerate}
\subsection{Schema-enabled decoding}
When in schema-enabled mode the basic parsing steps are essentially the same.
The difference is in the \texttt{EXIPSchema*} parameter passed to the \texttt{initParser()}.
You need to pass a valid \texttt{EXIPSchema} object and not NULL as in the schema-less
case. This object contains all the definitions and constrains from the XML schema and
its creation is a topic of section \ref{sec:Schema-Infromation} \textbf{Schema Information}.
During parsing, the value types as defined by the schema definitions
are delivered through the corresponding callback handlers and not only \texttt{stringData()}.
For example, \texttt{xsd:integer} is delivered through \texttt{intData()} and
\texttt{xsd:boolean} through \texttt{booleanData()}. Apart from that all other decoding steps
are the same as in schema-less mode.