\section{Schema Information}
\label{sec:Schema-Infromation}

This section examines different ways of constructing \texttt{EXIPSchema} object
containing the XML schema definitions and constrains. The header file \texttt{grammarGenerator.h}
defines a single function that takes a XML schema document(s) and converts them to 
\texttt{EXIPSchema} object. This function can be used to dynamically (at run-time)
parse a schema and generate the EXI grammar constructs for schema-enabled
parsing and serialization. As mentioned earlier, the EXIP library currently supports
only XML schema definitions represented in EXI format. Moreover, the fidelity option \texttt{Preserve.prefixes} must be
set in order to decode the QNames in the value items correctly (see the EXI specification for
more information on that).

When the XML schemes are static (only used at
compile time) the \texttt{grammarGen} module is not needed and can be excluded from the build.
In this case the \texttt{EXIPSchema} object can be build from automatically generated source code
by the \texttt{exipg} utility implemented in \texttt{utils/schemaHandling/createGrammars.c}.

\subsection{Using exipg utility}
The \texttt{exipg} utility is a command line tool for generating EXI grammar definitions
for schema-enabled EXI processing based on XML schemes.
It uses the grammar generation
function in \texttt{grammarGenerator.h} and as such also requires EXI encoded XML schemes.
There are three modes defining the output of the tool:
\begin{description}
 \item[exip] In this mode the XML schema is serialized into an EXIP specific format. The output is
	    an EXI document that later can be loaded into the EXIP library for schema-enabled processing.
	    This option is currently not implemented.
 \item[text] In this mode the grammar definitions are printed in human readable form. This mode is
	    useful for debugging purposes.
 \item[src] In this mode the grammar definitions are generated in \texttt{C} source code. The code
	    can use either static definitions (everything is in the programming memory) or
	    dynamic definitions (dynamic memory allocations). The dynamic definitions are
	    not supported in this release.
 \end{description}
As an example, the command line arguments used to generate the EXI Options document grammars are:
\begin{lstlisting}
exipg -src=static -pfx=ops_ -mask=1000000 EXIOptions-xsd.exi staticEXIOptions.c
\end{lstlisting}

\subsection{Converting XML Schema files to EXI}
Currently, there are no XML Schema editing tools that are capable of saving the document in EXI format.
For that reason it is required that you convert the text XML Schema to EXI encoding before using it
with EXIP. You can use any of the open source Java implementations of EXI for that purpose. Particularly
convenient for use is the \href{http://sourceforge.net/projects/exiprocessor/files/}{ExiProcessor} command line utility.
The correct arguments to be passed to the utility for converting the EXIOptions.xsd are:
\begin{lstlisting}
java -jar ExiProcessor.jar -header_options -preserve_prefixes
  -xml_in EXIOptions.xsd -exi_out EXIOptions-xsd.exi
\end{lstlisting}