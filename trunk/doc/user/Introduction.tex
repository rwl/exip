\section{Introduction}
\label{sec:Introduction}

EXIP is a free, open-source, C language implementation of the Efficient XML Interchange (EXI) standard for representing XML documents.
The EXI format keeps the data in a binary form and is a compact and efficient way for storing, processing and transmitting
XML structured information. Each EXI document consists of two parts: a header and a body. The header defines different
encoding/decoding parameters that affect the compactness, processing efficiency and memory usage when processing the
document (the EXI body part). For example, the document size can be further reduced by using the EXI compression option indicating the
application of DEFLATE algorithm for data compression. Another important parameter is connected to the use of XML
schema information that sets constraints on the structure and content of the document. The schema information must
be available before the EXI encoding/decoding, and it should be either statically set or communicated in advance
using XML schema languages such as W3C XML Schema. Schema-enabled EXI processing is much faster and results in
smaller EXI documents compared to schema-less processing. Furthermore, there are a set of options, called fidelity
options, determining the degree to which the EXI encoding preserves all the information items found in the XML
specification. More information about EXI is given in the W3C specification \cite{EXI1} and
primer \cite{Peintner2009a}. There are other specifications/formats for increasing the compactness and efficiency
of XML such as Fast Infoset, X.694 ASN.1 etc. Comparison with EXI and performance measurements are available from
the W3C EXI Working Group \cite{Bournez2009, White2007}.

\subsection{Intended audience}
\label{sec:audience}
This user guide is structured as a step-by-step tutorial on using the EXIP library. The guide is written with the
assumption that the reader has fairly good understanding of the C programming language, XML and XML Schema technologies.
While the goal was to give short introductory notes on the EXI format, the reader is expected to be familiar with the
description of the EXI header options provided in the specification. The intended audience of this guide are
programmers working with XML/EXI technologies for applications with high efficiency requirements such as networked
embedded systems, productivity tools and servers. Developers interested in contributing to the EXIP project in
terms of extra features, bug reports and patches are also encouraged to read it.

This guide does not assume the use of any concrete hardware/software platform. All the examples and instructions
should be valid for all execution environments.

\subsection{Organization}
\label{sec:organization}

The rest of this guide is organized in five sections. Section \ref{sec:Basic-Concepts} \textbf{Basic Concepts}
gives a high level overview of some important concepts for working with EXIP. Section \ref{sec:Maturity-Statement} \textbf{Maturity Statement}
describes the current status of the project. Sections \ref{sec:Serialization} \textbf{Serialization} and 
\ref{sec:Parsing} \textbf{Parsing} present the steps required for using the EXIP API for processing EXI streams.
The last part - \ref{sec:Schema-Infromation} \textbf{Schema Information} is devoted to working with XML Schema
definitions used for schema-enabled EXI parsing and serialization.

\subsection{Acknowledgment}
\label{sec:acknowledgment}
The author would like to thank the European Commission
and the partners of the EU FP7 project IMC-AESOP \href{http://www.imc-aesop.eu}{(www.imc-aesop.eu)}
for their support.